\documentclass[]{article}

%opening
\title{Project 1 Numerical Methods}
\author{Dustin O'Brien}

\begin{document}

\maketitle


\section*{Problem 3}
Consider the following function $f(x)$
\[ f(x) = \sqrt{x + 1} \]
Using Taylors Theorem we know
\[f(x) = P_n(x) + R_n(x)\]

where $P_n(x)$ and $R_n(x)$ are defined as

\[P_n(x) = \sum_{k=0}^{n} \frac{(x-x_0)^k}{k!} f^{(k)}(x_0)\]
\[R_n(x) = \frac{(x - x_0)^{n+1}}{(n+1)!} f^{(n+1)}(\xi_x)\]
where $ x> \xi_x >x_0$ \\\\
Given $x_0 = 0$ and $n = 1$ these formulas simplify to

\[P_1(x) = \sum_{k=0}^{1} \frac{x^k}{k!} f^{(k)}(0) = \frac{x^0}{0!} f(0) + \frac{x^{1}}{1!}f'(0) = f(0) + xf'(0)\]
\[R_1(x) = \frac{x^{2}}{2!} f^{''}(\xi_x) = \frac{x^{2}}{2}f''(\xi_x)\]
Where $x > \xi_x > 0$
\\\\
Recall that f(x) = $\sqrt{x+1}$


\[	f(x) = \sqrt{x+1} \]
\[	f'(x) = \frac{1}{2\sqrt{x+1}} \]
\[	f''(x) = -\frac{1}{4\sqrt{(x+1)^3}} \]

This means
\[ P_1(x) = \sqrt{0 + 1} + \frac{x}{2\sqrt{0+1}} = 1 + \frac{x}{2}\]
\[R_1(x) = \frac{x^2}{2} (-\frac{1}{4\sqrt{(\xi_x+1)^3}}) = -\frac{x^2}{8\sqrt{(\xi_x+1)^3}} = O(\frac{x^2}{\sqrt{(\xi_x +1)^3}})\]

Since $\frac{1}{\sqrt{(\xi_x + 1)^3}}$ is a decreasing function from $(-1, \infty)$ and $\xi_x \in [0, \infty)$ the maximum of this function is at $\xi_x = 0$ and therefore bounded at that point

\[\frac{1}{\sqrt{(0 + 1)^3}} = 1 \]

\[O(x^2 * 1) = O(\frac{x^2}{\sqrt{(\xi_x +1)^3}})\]

\[O(x^2) = O(x^2 * 1)\]

Therefore, by plugging into Taylors Theroem we get
\[\sqrt{x+1} = 1 + \frac{x}{2}+O(x^2)\]



\section*{Problem 4}

Consider the following function $f(x)$
\[ f(x) = \frac{1}{x + 1} \]

Using Taylors Theorem with $x_0 = 0$ and $n = 2$

\[P_2(x) = \sum_{k=0}^{2} \frac{x^k}{k!} f^{(k)}(0) = \frac{x^0}{0!} f(0) + \frac{x^{1}}{1!}f'(0) + \frac{x^2}{2!}f''(0)= f(0) + xf'(0) + \frac{x^2}{2}f''(0)\]
\[R_2(x) = {\frac{x^3}{6}f'''(\xi_x)}\]

Using $f(x) = \frac{1}{x+1}$ we know

\[f(x) = \frac{1}{x+1}\]
\[f'(x) = -\frac{1}{(x+1)^2}\]
\[f''(x) = \frac{2}{(x+1)^3}\]
\[f'''(x) = -\frac{6}{(x+1)^4}\]

Using this we know

\[P_2(x) = 1 - x + \frac{x^2}{2}*2 = 1 - x + {x^2}\]
\[R_2(x) = \frac{x^3}{6} * -\frac{6}{(\xi_x+1)^4} = O(\frac{x^3}{6} * -\frac{6}{(\xi_x+1)^4}) = O(\frac{x^3}{(\xi_x+1)^4})\]

As, $ O(\frac{1}{(\xi_x+1)^4})$ is an decreasing function from $(-1 , \infty)$ and $\xi_x \in [0 ,\infty)$ the maximum value is at $\xi_x = 0$

\[\frac{1}{(0+1)^4} = 1\]
\[O(x^3) = O(x^3 * 1) = O(\frac{x^3}{(\xi_x+1)^4})\]

Therefore, using Taylors Theroem we know

\[\frac{1}{x+1} =  1 - x + x^2 + O(x^3)\]



\section*{Problem 5}

Consider the following function $f(x)$
\[ f(x) = \sin{(x)} \]
Using Taylors Theorem with $x_0 = 0$ and $n = 0$
	\[f(x) = P_0(x) + R_0(x)\]
	\[P_0(x)=\frac{x^0}{0!}f(0)=f(0)=\sin(0)=0\]
	\[R_0(x)=\frac{(x-0)^{1}}{(0+1)!}f'(\xi_x) = xf'(\xi_x)\]
Using the original statement we know
\[f(x) = \sin(x)\]	
\[f'(x) = \cos(x)\]	
We then know,
\[R_0(x) = x \cos(\xi_x)\]
\[R_0(x) = O(x \cos(\xi_x))\]
Since the function $\cos(x)$ is bounded between [-1,1] the function $R_0(x)$ can be represented as

\[R_0(x)=O(x * 1) = O(x)\]

Since, we know for $ \forall x > 1, x^3 > x$ This can then be used to say $O(x^3) > O(x)$

This means \[R_0(x) = O(x^3)\]

Therefore,

\[\sin(x) = 1 + O(x^3)\]

\section*{Problem 6}

Consider the following function $f(x)$
\[f(x) = \sum_{k=0}^{n}r^k\]
Using the given sumation formula we know
\[f(x) = \sum_{k=0}^{n}r^k = \frac{1-r^{n+1}}{1-r}=\frac{O(1-r^{n+1})}{1-r}\]

Since we know the $\max(1, -r^{n+1}) = -r^{n+1}$ we can use definition of $O()$ to remove 1 meaning

\[\frac{O(1-r^{n+1})}{1-r} = \frac{O(-r^{n+1})}{1-r}\]

Since, $O()$ allows for $\forall $ coefficient $C$ multiplying current C by -1 will allow us to remove coefficeints Therefore,

\[\frac{O(-r^{n+1})}{1-r} = \frac{O(r^{n+1})}{1-r}\]

\section*{Problem 15}

\end{document}
