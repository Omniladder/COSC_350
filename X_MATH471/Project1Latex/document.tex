\documentclass[]{article}

%opening
\title{Project 1 Numerical Methods}
\author{Dustin O'Brien}

\begin{document}

\maketitle


\section*{Problem 3}

	Consider the function $f_x(x)$ where
	\[f(x) = \sqrt{1 + x^2}\]
	Using Taylors Theorem we know
	\[P_n(x) = \sum_{k=0}^{n}\frac{x^k}{k!}f^{(k)}(0)\]
	
	Let n = 6 then

	\[P_6(x) = \sum_{k=0}^{6}\frac{x^k}{k!}f^{(k)}(0)\]
	
	Using the defined function we know the following
	
	\[f(x) = \sqrt{1 + x^2} = (1 + x^2)^{\frac{1}{2}}\]
	\[f'(x) = x(1 + x^2)^{-\frac{1}{2}}\]
	\[f''(x) = (1 + x^2)^{-\frac{1}{2}} - x^2(1 + x^2)^{-\frac{3}{2}}\]
	\[f'''(x) = -3x(1 + x^2)^{-\frac{3}{2}} + 3x^3(1 + x^2)^{-\frac{5}{2}}\]
	\[f^{(4)}(x) = -3(1 + x^2)^{-\frac{3}{2}} + 18x^2(1 + x^2)^{-\frac{5}{2}} - 15x^4(1 + x^2)^{-\frac{7}{2}}\]
	\[f^{(5)}(x) = 45x(1 + x^2)^{-\frac{5}{2}} -150x^3(1 + x^2)^{-\frac{5}{2}} - 1056x^4(1 + x^2)^{-\frac{7}{2}}\]
	\[f^{(6)}(x) = 45(1 + x^2)^{-\frac{5}{2}} - 675x^2(1+x^2)^{-\frac{7}{2}} + 1575x^4(1+x^2)^{-\frac{9}{2}} - 945x^6(1+x^2)^{-\frac{11}{2}}\]
	And $P_6(x)$ expands too
	\[P_6(x) = \frac{x^0}{0!}f(0)+\frac{x^1}{1!}f'(0)+\frac{x^2}{2!}f''(0)+\frac{x^3}{3!}f'''(0)+\frac{x^4}{4!}f^{(4)}(0)+\frac{x^5}{5!}f^{(5)}(0)+\frac{x^6}{6!}f^{(6)}(0)\]
	\[P_6(x) = f(0)+f'(0)+\frac{x^2}{2}f''(0)+\frac{x^3}{6}f'''(0)+\frac{x^4}{24}f^{(4)}(0)+\frac{x^5}{120}f^{(5)}(0)+\frac{x^6}{720}f^{(6)}(0)\]
	Using the previously derived equations this can then be turned into
	\[P_6(x) = 1+\frac{x^2}{2}+\frac{x^4}{24}(-3)+\frac{x^6}{720}45\]
	\[P_6(x) = 1+\frac{x^2}{2}-\frac{x^4}{8}+\frac{x^6}{16}\]
	There fore the Taylor Series Polynomial is
	\[P_6(x) = 1+\frac{x^2}{2}-\frac{x^4}{8}+\frac{x^6}{16}\]
	
	\section*{Problem 4}
	Given $R(x) = \frac{|x|^6}{6!}e^\xi$ where $0 < \xi < x$ and $x \in [-1,1]$ find upperbound of $|R(x)|$
	
	\[R(x) = \frac{|x|^6}{6!}e^\xi\]
	\[|R(x)| = |\frac{|x|^6}{6!}e^\xi|\]
	
	since the max value of $|x|$ can only be 1 and $\frac{x}{6!}$ is an increasing function the $|R(x)| \le |\frac{|1|^6}{6!}e^\xi|$
	
	\[|R(x)| \le |\frac{|1|^6}{6!}e^\xi|\]
	\[|R(x)| \le |\frac{1}{720}e^\xi|\]
	
	And since the max value of $\xi$ is also 1 and $e^\xi$ is also an increasing function $|R(x)| \le |\frac{1}{720}e^1|$
	
	\[|R(x)| \le |\frac{1}{720}e^1|\]
	\[|R(x)| \le |\frac{e}{720}|\]
	
	\section*{Problem 5}
	
		Given $R(x) = \frac{|x|^6}{6!}e^\xi$ where $0 < \xi < x$ and $x \in [-\frac{1}{2},\frac{1}{2}]$ find upperbound of $|R(x)|$
	
	\[R(x) = \frac{|x|^6}{6!}e^\xi\]
	\[|R(x)| = |\frac{|x|^6}{6!}e^\xi|\]
	
	since the max value of $|x|$ can only be $\frac{1}{2}$ and $\frac{x}{6!}$ is an increasing function the $|R(x)| \le |\frac{|\frac{1}{2}|^6}{6!}e^\xi|$
	
	\[|R(x)| \le |\frac{|\frac{1}{2}|^6}{6!}e^\xi|\]
	\[|R(x)| \le |\frac{\frac{1}{64}}{720}e^\xi|\]
	\[|R(x)| \le |\frac{1}{16384}e^\xi|\]
	
	And since the max value of $\xi$ is also $\frac{1}{2}$ and $e^\xi$ is also an increasing function $|R(x)| \le |\frac{1}{16384}e^{\frac{1}{2}}|$
	
	\[|R(x)| \le |\frac{1}{16384}e^{\frac{1}{2}}|\]
	\[|R(x)| \le |\frac{e^{\frac{1}{2}}}{16384}|\]
	
	Therefore $|R(x)|$ is bounded by $\frac{e^{\frac{1}{2}}}{16384}$
	
	
		\section*{Problem 6}
	
	Given $R(x) = \frac{|x|^4}{4!}e^\xi$ where $0 < \xi < x$ and $x \in [-\frac{1}{2},\frac{1}{2}]$ find upperbound of $|R(x)|$
	
	\[R(x) = \frac{|x|^4}{4!}e^\xi\]
	\[|R(x)| = |\frac{|x|^4}{4!}e^\xi|\]
	
	since the max value of $|x|$ can only be $\frac{1}{2}$ and $\frac{x}{4!}$ is an increasing function the $|R(x)| \le |\frac{|\frac{1}{2}|^4}{4!}e^\xi|$
	
	\[|R(x)| \le |\frac{|\frac{1}{2}|^4}{4!}e^\xi|\]
	\[|R(x)| \le |\frac{\frac{1}{16}}{24}e^\xi|\]
	\[|R(x)| \le |\frac{1}{384}e^\xi|\]
	
	And since the max value of $\xi$ is also $\frac{1}{2}$ and $e^\xi$ is also an increasing function $|R(x)| \le |\frac{1}{384}e^{\frac{1}{2}}|$
	
	\[|R(x)| \le |\frac{1}{384}e^{\frac{1}{2}}|\]
	\[|R(x)| \le |\frac{e^{\frac{1}{2}}}{384}|\]
	
	Therefore $|R(x)|$ is bounded by $\frac{e^{\frac{1}{2}}}{384}$
	
	\section*{Problem 15}
	
	Given The statement \[\arctan(\frac{1}{239}) = 4\arctan(\frac{1}{5}) - \arctan(1)\]
	derive formula for $\pi$ and Find The order of polynomial s.t. The errors is less than $10^{-100}$ and $10^{-1000}$
	
	\[\arctan(\frac{1}{239}) = 4\arctan(\frac{1}{5}) - \arctan(1)\]
	\[ \arctan(1) = 4\arctan(\frac{1}{5}) - \arctan(\frac{1}{239})\]
	\[ \frac{\pi}{4} = 4\arctan(\frac{1}{5}) - \arctan(\frac{1}{239})\]
	\[ \pi = 16\arctan(\frac{1}{5}) - 4\arctan(\frac{1}{239})\]
	
	Using the Gregory Series we know
	\[\arctan(x) = \sum_{k=0}^{n}((-1)^k\frac{x^{2k+1}}{(2k+1)})+\int_{0}^{x}\frac{t^{2n+2}}{1+t^2}dt\]
	For simplicity Let their be a function $P_n(x)$ s.t.
	
	\[P_n(x) = \sum_{k=0}^{n}((-1)^k\frac{x^{2k+1}}{(2k+1)})\]
	Then,
	\[\arctan(x) = P_n(x)+\int_{0}^{x}\frac{t^{2n+2}}{1+t^2}dt\]
	Since $1+t^2$ is always positive and always greater than 1 as long as $t \ge 0$ than it will as long as $x \ge 0$ 
	
	\[\arctan(x) \le P_n(x)+\int_{0}^{x}t^{2n+2}dt\]
	
	\[\arctan(x) \le P_n(x)+\frac{x^{2n+3}}{2n+3}\]
	
	Then,
	\[ \pi \le 16(P_n(\frac{1}{5})+\frac{1}{5^{2n+3}(2n+3)}) - 4(P_n(\frac{1}{239})+\frac{1}{239^{2n+3}(2n+3)}))\]
	
	\[ \pi \le 16P_n(\frac{1}{5})+\frac{16}{5^{2n+3}(2n+3)} - 4P_n(\frac{1}{239})+\frac{4}{239^{2n+3}(2n+3)})\]
	
	\[ \pi - 16P_n(\frac{1}{5}) + 4P_n(\frac{1}{239}) \le \frac{16}{5^{2n+3}(2n+3)} +\frac{4}{239^{2n+3}(2n+3)})\]
	
	\[ \pi - 16P_n(\frac{1}{5}) + 4P_n(\frac{1}{239}) \le |\frac{16}{5^{2n+3}(2n+3)} +\frac{4}{239^{2n+3}(2n+3)})|\]
	
	Let there exist another function $R_n(x)$ s.t.
	
	\[R_n(x) = |\frac{16}{5^{2n+3}(2n+3)} +\frac{4}{239^{2n+3}(2n+3)})|\]
	 
	\[ \pi - 16P_n(\frac{1}{5}) + 4P_n(\frac{1}{239}) \le R_n(x)\]
	
	\[ \pi - 16P_{71}(\frac{1}{5}) + 4P_{71}(\frac{1}{239}) \le R_{71}(x) \le 10^{-101}\]
	
	\[ \pi - 16P_{714}(\frac{1}{5}) + 4P_{714}(\frac{1}{239}) \le R_{714}(x) \le 10^{-1001}\]
	
	Therefore 71 terms are required to have 100 digits of accuracy and roughlt 714 are needed for 1000 digits of accuracy

\end{document}
